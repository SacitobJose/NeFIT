% Setup -------------------------------

\documentclass[a4paper]{report}
\usepackage[a4paper, total={6in, 10in}]{geometry}
\setcounter{secnumdepth}{3}
\setcounter{tocdepth}{3}

\PassOptionsToPackage{hyphens}{url}
\usepackage{hyperref}

% Encoding
%--------------------------------------
\usepackage[T1]{fontenc}
\usepackage[utf8]{inputenc}
%--------------------------------------

% Portuguese-specific commands
%--------------------------------------
\usepackage[portuguese]{babel}
%--------------------------------------

% Hyphenation rules
%--------------------------------------
\usepackage{hyphenat}
%--------------------------------------

% Capa do relatório

\title{
	Paradigmas de Sistemas Distribuídos
	\\ \Large{\textbf{Trabalho Prático}}
	\\ -
	\\ Mestrado em Engenharia Informática
	\\ \large{Universidade do Minho}
	\\ Relatório
}
\author{
	\begin{tabular}{ll}
		\textbf{Grupo}
		\\\hline
		PG41080 & João Ribeiro Imperadeiro
		\\
		PG41081 & José Alberto Martins Boticas
		\\
		PG41091 & Nelson José Dias Teixeira
	\end{tabular}
}

\date{\today}

\begin{document}

\begin{titlepage}
    \maketitle
\end{titlepage}

% Índice

\tableofcontents

% Introdução

\chapter{Introdução} \label{intro} {
	Neste trabalho prático é requerido o desenvolvimento de um protótipo de uma plataforma de negociação entre fabricantes e importadores de produtos. 
	Este protótipo é composto por cliente, servidor de \textit{front-end}, negociador e catálogo, de entidades e negociações em curso. Os clientes podem existir em elevado número, sendo que 
	cada um deles desempenhará sempre ou o papel de fabricante ou de importador. Cada fabricante indicará a disponibilidade para produzir um determinado artigo, numa quantidade mínima e máxima, 
	a um preço mínimo (unitário), bem como o período de tempo durante o qual os importadores poderão fazer ofertas de encomenda (período de negociação). Por sua vez, cada importador indica a quantidade 
	e valor unitário a que está disposto a pagar por um determinado artigo de um fabricante. Os clientes autenticam-se no servidor de \textit{front-end}, o qual encaminha as suas ordens para um (de entre vários) 
	negociadores. O catálogo disponibilizará uma interface \textit{RESTful}, que permitirá obter informação sobre os fabricantes, importadores, e negociações em curso. Como tal, por forma a implementar este protótipo, 
	foi utilizada a linguagem de programação \textit{Java} (cliente, negociador e catálogo), \textit{Erlang} (servidor \textit{front-end}), e, ainda, \textit{Protocol Buffers}, \textit{ZeroMQ} e \textit{Dropwizard}.
}

\chapter{O sistema}
	\section{Funcionamento}

	O sistema é dividido em quatro partes principais: servidor \textit{front-end}, cliente, negociador e catálogo.

	O cliente é a entidade que representa importadores ou fabricantes. 
	O mesmo apresenta uma interface rudimentar que permite ao utilizador fazer ofertas ou encomendas, dependendo se é um importador ou fabricante.
	
	O servidor, por sua vez, é responsável por receber estas encomendas e fazê-las chegar ao negociador correspondente. Tem ainda um pequeno mecanismo de autenticação para os seus clientes.
	Foi ainda tomada a decisão de que cada produto seria atribuído a um negociador, sendo esse negociador o responsável por todas as negociações que existam com esse produto.
	
	Um negociador é responsável por intermediar as negociações entre importadores e fabricantes, fazendo corresponder ofertas e encomendas que satisfaçam os requisitos de ambas.

	% FALAR DO CATÁLOGO

	\section{Implementação}
	
	\subsection{Servidor \textit{front-end}}
	A implementação do servidor foi feita em Erlang. O servidor divide-se em diversos ficheiros, cada um dos quais corresponde a um tipo de ator, que se dividem em seis tipos:
	\begin{itemize}
		\item Servidor;
		\item \texttt{loginHandler};
		\item Clientes;
		\item Importadorers;
		\item Fabricantes;
		\item negotiatorsHandler;
		\item Negociadores.
	\end{itemize}

	\subsubsection{Servidor central}
	Este é o primeiro ator criado, responsável pela criação de todos os outros atores e por aceitar novas conexões (clientes), bem como pela conexão inicial de todos os negociadores.

	\subsubsection{\texttt{loginHandler}}
	Este ator é registado e criado pelo anterior. É o responsável pela autenticação e registo dos clientes, guardando todas as informações nesse sentido.

	\subsubsection{Clientes}
	Ator criado a cada nova conexão. Espera a receção de uma comunicação TCP com a autenticação do cliente e usando o \texttt{loginHandler} para confirmar a sua identificação.
	Segue-se a criação de novo ator, importador ou fabricante, que fica responsável pela comunicação com esse cliente. O ator atual é substituído pelo novo.

	\subsubsection{Importadores e Produtores}
	Ator criado pelo anterior, sempre que um cliente, do tipo importador ou fabricante, se autentica com sucesso. 
	Fica então responsável pela comunicação com o cliente, recebendo as suas encomendas/ofertas e fazendo com que as mesmas cheguem ao respetivo negociador, através do \texttt{negotiatorsHandler}.

	\subsubsection{\texttt{negotiatorsHandler}}
	O \texttt{negotiatorsHandler} é um ator, registado pelo servidor central, que é contactado por importadores ou fabricantes que queiram fazer encomendas ou ofertas. 
	Ao recebê-las, consoante o produto, passa-as para o respetivo negociador. Cada novo produto é atribuído a um negociador, em estilo \textit{round-robin}.

	\subsubsection{Negociadores}
	Existirá um ator deste tipo para cada negociador (quantidade pré-definida). Um negociador é responsável por um certo número de produtos, tratando de todas as ofertas e encomendas desse produto.
	Quando um cliente faz uma oferta ou encomenda, a mesma acaba por chegar ao ator (negociador) responsável pelo produto, fazendo-a depois chegar ao negociador propriamente dito (exterior ao servidor).


	\subsection{Cliente}


	\subsection{Negociador}


	\subsection{Catálogo}


\chapter{Conclusão}

\chapter{Webgrafia}
	\begin{itemize}
		\item \textbf{\textit{Protocol Buffers}}:
		\par \textit{\url{https://developers.google.com/protocol-buffers}}
        \item \textbf{\textit{Protocol Buffers - Java}}:
		\par \textit{\url{https://developers.google.com/protocol-buffers/docs/reference/java-generated}}
		\item \textbf{Documentação - \textit{Java}}:
		\par \textit{\url{https://docs.oracle.com/en/java/javase/11/docs/api/index.html}}
    \end{itemize}

\end{document}