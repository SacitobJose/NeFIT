% Setup -------------------------------

\documentclass[a4paper]{report}
\usepackage[a4paper, total={6in, 10in}]{geometry}
\setcounter{secnumdepth}{3}
\setcounter{tocdepth}{3}

\usepackage{xcolor}

% Encoding
%--------------------------------------
\usepackage[T1]{fontenc}
\usepackage[utf8]{inputenc}
%--------------------------------------

% Portuguese-specific commands
%--------------------------------------
\usepackage[portuguese]{babel}
%--------------------------------------

% Hyphenation rules
%--------------------------------------
\usepackage{hyphenat}
%--------------------------------------

% Capa do relatório

\title{
	Paradigmas de Sistemas Distribuídos
	\\ \Large{\textbf{Trabalho Prático}}
	\\ -
	\\ Mestrado em Engenharia Informática
	\\ \large{Universidade do Minho}
	\\ Relatório
}
\author{
	\begin{tabular}{ll}
		\textbf{Grupo}
		\\\hline
		PG41080 & João Ribeiro Imperadeiro
		\\
		PG41081 & José Alberto Martins Boticas
		\\
		PG41091 & Nelson José Dias Teixeira
	\end{tabular}
}

\date{\today}

\begin{document}

\begin{titlepage}
    \maketitle
\end{titlepage}

% Resumo

\begin{abstract}
	Este projeto de avaliação relativo à unidade curricular de Fundamentos de Sistemas Distribuídos consiste, globalmente, ...
\end{abstract}

% Índice

\tableofcontents

% Introdução

\chapter{Introdução} \label{intro} {
	\section{Contextualização}
}

\chapter{Metodologia}
    \section{Modelo adoptado}

\chapter{Resultados}

\chapter{Conclusão}

\chapter{Webgrafia}
    \begin{itemize}
        \item \textit{Website}:
        \par \textit{https://www.google.com}
    \end{itemize}

\appendix
\chapter{Observações}

\end{document}