% Setup -------------------------------

\documentclass[a4paper]{report}
\usepackage[a4paper, total={6in, 10in}]{geometry}
\setcounter{secnumdepth}{3}
\setcounter{tocdepth}{3}

\PassOptionsToPackage{hyphens}{url}
\usepackage{hyperref}

% Encoding
%--------------------------------------
\usepackage[T1]{fontenc}
\usepackage[utf8]{inputenc}
%--------------------------------------

% Portuguese-specific commands
%--------------------------------------
\usepackage[portuguese]{babel}
%--------------------------------------

% Hyphenation rules
%--------------------------------------
\usepackage{hyphenat}
%--------------------------------------

% Capa do relatório

\title{
	Paradigmas de Sistemas Distribuídos
	\\ \Large{\textbf{Trabalho Prático}}
	\\ -
	\\ Mestrado em Engenharia Informática
	\\ \large{Universidade do Minho}
	\\ Relatório
}
\author{
	\begin{tabular}{ll}
		\textbf{Grupo}
		\\\hline
		PG41080 & João Ribeiro Imperadeiro
		\\
		PG41081 & José Alberto Martins Boticas
		\\
		PG41091 & Nelson José Dias Teixeira
	\end{tabular}
}

\date{\today}

\begin{document}

\begin{titlepage}
    \maketitle
\end{titlepage}

% Índice

\tableofcontents

% Introdução

\chapter{Introdução} \label{intro} {
	Neste trabalho prático é requerido o desenvolvimento de um protótipo de uma plataforma de negociação entre fabricantes e importadores de produtos. 
	Este protótipo é composto por cliente, servidor de \textit{front-end}, negociador e catálogo, de entidades e negociações em curso. Os clientes podem existir em elevado número, sendo que 
	cada um deles desempenhará sempre ou o papel de fabricante ou de importador. Cada fabricante indicará a disponibilidade para produzir um determinado artigo, numa quantidade mínima e máxima, 
	a um preço mínimo (unitário), bem como o período de tempo durante o qual os importadores poderão fazer ofertas de encomenda (período de negociação). Por sua vez, cada importador indica a quantidade 
	e valor unitário a que está disposto a pagar por um determinado artigo de um fabricante. Os clientes autenticam-se no servidor de \textit{front-end}, o qual encaminha as suas ordens para um (de entre vários) 
	negociadores. O catálogo disponibilizará uma interface \textit{RESTful}, que permitirá obter informação sobre os fabricantes, importadores, e negociações em curso. Como tal, por forma a implementar este protótipo, 
	foi utilizada a linguagem de programação \textit{Java} (cliente, negociador e catálogo), \textit{Erlang} (servidor \textit{front-end}), e, ainda, \textit{Protocol Buffers}, \textit{ZeroMQ} e \textit{Dropwizard}.
}

\chapter{O sistema}
	\section{Funcionamento}

	\section{Implementação}
	
	\subsection{Servidor}
	A implementação do servidor foi feita em \textit{Erlang}. O servidor divide-se em diversos ficheiros, cada um dos quais corresponde a um tipo de ator, que se dividem em seis tipos:
	\begin{itemize}
		\item \texttt{server.erl};
		\item \texttt{login.erl};
		\item \texttt{client.erl};
		\item \texttt{importer.erl};
		\item \texttt{producer.erl};
		\item \texttt{negotiator.erl}.
	\end{itemize}

	\subsubsection{\texttt{server.erl}}
	Este é o primeiro ator criado, responsável pela criação de todos os outros atores e por aceitar novas conexões (clientes).

	\subsubsection{\texttt{login.erl}}
	Este tipo de ator é registado, como \texttt{loginHandler}, e criado pelo anterior. É o responsável pela autenticação e registo dos clientes, guardando todas as informações nesse sentido.

	\subsubsection{\texttt{client.erl}}
	Ator criado a cada nova conexão. Espera a receção de uma comunicação \textit{TCP} com a autenticação do cliente e usando o \texttt{loginHandler} para confirmar a sua identificação.
	Segue-se a criação de novo ator, importador ou fabricante, que fica responsável pela comunicação com esse cliente. O ator atual é substituído pelo novo.

	\subsubsection{\texttt{importer.erl}}
	Ator criado pelo anterior, sempre que um cliente, do tipo importador, se autentica com sucesso. 
	Fica então responsável pela comunicação com o importador, recebendo as suas encomendas e fazendo com que as mesmas cheguem ao respetivo negociador, através do \texttt{negotiatorsHandler}.

	\subsubsection{\texttt{producer.erl}}
	Semelhante ao anterior, mas para um cliente do tipo fabricante.

	\subsubsection{\texttt{negotiator.erl}}
	Existirá um ator deste tipo para cada negociador (quantidade pré-definida).

	\subsection{Cliente}
	Tal como pedido nos requisitos do trabalho, a implementação do cliente foi feita em \textit{Java}. Para além de todas as classes nativas, foram ainda usadas funcionalidades das bibliotecas \textit{ZeroMQ}, uma implementação em \textit{Java} da ferramenta \textit{ZeroMQ} para envio/receção de mensagens assíncronas, e \textit{Protocol Buffers}, um formato de serialização de dados desenvolvido pela \textit{Google}. Posto isto, passaremos ao detalhe do funcionamento do cliente.
	
	O cliente, aquando do início da sua utilização por parte de um utilizador, começa por pedir os dados de autenticação e entra em contacto com o servidor para verificar a sua validade. Em caso afirmativo, é criada uma \textit{thread} adicional que recebe informações do servidor e o utilizador é enviado para um de dois menus: o de fornecedor ou de importador. Se o utilizador for um importador, é lançada ainda outra \textit{thread}, responsável por receber updates do catálogo quanto a produtores que o utilizador subscreveu.
	
	De seguida, após toda esta configuração inicial, é mostrado um dos seguintes menus com as operações indicadas:
	
	\begin{enumerate}
		\item \textbf{Importador}:
		\begin{itemize}
			\item Oferta de encomenda;
			\item Subscrever notificações;
			\item Cancelar notificações.
			\item Atualizações.
		\end{itemize}

		\item \textbf{Fornecedor}:
		\begin{itemize}
			\item Oferta de produção;
			\item Atualizações.
		\end{itemize}
	\end{enumerate}

	Começando pelas operações do importador, temos que este pode encomendar um produto, tendo para isso de indicar o nome do produtor/produto, o número de unidades e o preço que está disposto a pagar por unidade. Pode ainda subscrever ou cancelar a subscrição de atualizações sobre um determinado produtor, bastando indicar o nome desse produtor.
	
	Passando ao fornecedor, este pode apenas colocar uma oferta de produção, indicando o nome do produto, quantidade mínima/máxima, preço mínimo por unidade e período de negociação (em segundos).
	
	Quanto às mensagens do servidor e do catálogo, estas são apresentadas sempre que há informações para mostrar, tendo em conta que o utilizador não está a realizar nenhuma operação ou quando este clica na opção "Atualizações".
	
	Em ambos os casos, é dada a opção ao utilizador para sair da interface.
\chapter{Conclusão}

\chapter{Webgrafia}
	\begin{itemize}
		\item \textbf{\textit{Protocol Buffers}}:
		\par \textit{\url{https://developers.google.com/protocol-buffers}}
        \item \textbf{\textit{Protocol Buffers - Java}}:
		\par \textit{\url{https://developers.google.com/protocol-buffers/docs/reference/java-generated}}
		\item \textbf{Documentação - \textit{Java}}:
		\par \textit{\url{https://docs.oracle.com/en/java/javase/11/docs/api/index.html}}
    \end{itemize}

\end{document}